% Options for packages loaded elsewhere
\PassOptionsToPackage{unicode}{hyperref}
\PassOptionsToPackage{hyphens}{url}
%
\documentclass[
]{book}
\usepackage{amsmath,amssymb}
\usepackage{lmodern}
\usepackage{iftex}
\ifPDFTeX
  \usepackage[T1]{fontenc}
  \usepackage[utf8]{inputenc}
  \usepackage{textcomp} % provide euro and other symbols
\else % if luatex or xetex
  \usepackage{unicode-math}
  \defaultfontfeatures{Scale=MatchLowercase}
  \defaultfontfeatures[\rmfamily]{Ligatures=TeX,Scale=1}
\fi
% Use upquote if available, for straight quotes in verbatim environments
\IfFileExists{upquote.sty}{\usepackage{upquote}}{}
\IfFileExists{microtype.sty}{% use microtype if available
  \usepackage[]{microtype}
  \UseMicrotypeSet[protrusion]{basicmath} % disable protrusion for tt fonts
}{}
\makeatletter
\@ifundefined{KOMAClassName}{% if non-KOMA class
  \IfFileExists{parskip.sty}{%
    \usepackage{parskip}
  }{% else
    \setlength{\parindent}{0pt}
    \setlength{\parskip}{6pt plus 2pt minus 1pt}}
}{% if KOMA class
  \KOMAoptions{parskip=half}}
\makeatother
\usepackage{xcolor}
\usepackage{longtable,booktabs,array}
\usepackage{calc} % for calculating minipage widths
% Correct order of tables after \paragraph or \subparagraph
\usepackage{etoolbox}
\makeatletter
\patchcmd\longtable{\par}{\if@noskipsec\mbox{}\fi\par}{}{}
\makeatother
% Allow footnotes in longtable head/foot
\IfFileExists{footnotehyper.sty}{\usepackage{footnotehyper}}{\usepackage{footnote}}
\makesavenoteenv{longtable}
\usepackage{graphicx}
\makeatletter
\def\maxwidth{\ifdim\Gin@nat@width>\linewidth\linewidth\else\Gin@nat@width\fi}
\def\maxheight{\ifdim\Gin@nat@height>\textheight\textheight\else\Gin@nat@height\fi}
\makeatother
% Scale images if necessary, so that they will not overflow the page
% margins by default, and it is still possible to overwrite the defaults
% using explicit options in \includegraphics[width, height, ...]{}
\setkeys{Gin}{width=\maxwidth,height=\maxheight,keepaspectratio}
% Set default figure placement to htbp
\makeatletter
\def\fps@figure{htbp}
\makeatother
\setlength{\emergencystretch}{3em} % prevent overfull lines
\providecommand{\tightlist}{%
  \setlength{\itemsep}{0pt}\setlength{\parskip}{0pt}}
\setcounter{secnumdepth}{5}
\usepackage{booktabs}
\ifLuaTeX
  \usepackage{selnolig}  % disable illegal ligatures
\fi
\usepackage[]{natbib}
\bibliographystyle{apalike}
\IfFileExists{bookmark.sty}{\usepackage{bookmark}}{\usepackage{hyperref}}
\IfFileExists{xurl.sty}{\usepackage{xurl}}{} % add URL line breaks if available
\urlstyle{same} % disable monospaced font for URLs
\hypersetup{
  pdftitle={Baseline risk in medical decision making},
  pdfauthor={Alexandros Rekkas},
  hidelinks,
  pdfcreator={LaTeX via pandoc}}

\title{Baseline risk in medical decision making}
\usepackage{etoolbox}
\makeatletter
\providecommand{\subtitle}[1]{% add subtitle to \maketitle
  \apptocmd{\@title}{\par {\large #1 \par}}{}{}
}
\makeatother
\subtitle{From outcome prediction to the assessment of treatment effect heterogeneity}
\author{Alexandros Rekkas}
\date{}

\begin{document}
\maketitle

{
\setcounter{tocdepth}{1}
\tableofcontents
}
\hypertarget{introduction}{%
\chapter{Introduction}\label{introduction}}

Baseline risk is a crucial component of medical decision making. Because it
provides personalized quantification of the likelihood for unwanted events, it
is often used to guide treatment recommendations. For example in the European
Society of Cardiology and the European Society of Hypertension guidelines of
2018 for the management of arterial hypertension, treatment initiation is
based---among other things---on the patient's baseline 10-year cardiovascular
risk {[}\url{https://doi.org/10.1093/eurheartj/ehy339}{]}. Similarly, an algorithm for the
treatment of osteoporosis has been suggested, based on the stratification of
patients on their 10-year hip or major osteoporotic fracture risk
{[}\url{https://doi.org/10.1007/s00198-019-05176-3}{]}.

Best practices for developing prediction models, evaluating their performance,
and guiding their application in practice have been central focus of
methodological research {[}refs{]}. Usually, prediction models are evaluated on
their discriminative performance, i.e.~their ability to separate lower from
higher risk patients, and their calibration, i.e.~the agreement of predicted
risk to observed event rates {[}Ewout book{]}. These measures---though they are
useful for assessing the predictive performance of one or multiple candidate
prediction models---are not informative when it comes to applying these models
in practice. Baseline risk is only one of the crucial pieces required for
predicting individual responses to treatment. Knowledge of the patients'
responsiveness to treatment, their vulnerability to side-effects and their
utilities for other relevant outcomes is necessary information required for
making truly informed clinical decisions {[}Kravitz, 2004{]}. Our aim was to explore
approaches that incorporate baseline risk as the basis for medical decision
making, shifting the focus from outcome prediction to the evaluation of
treatment effect heterogeneity.

In order to provide the most optimal medical care, doctors are advised to align
their clinical practice with the results of well-conducted clinical trials, or
the aggregated results from multiple such trials {[}Greenfield 2007{]}. This
approach implicitly assumes that all patients eligible for treatment experience
the same effects (benefits and harms) of treatment as the reference trial
population. However, the estimated treatment effect is often an average of
heterogeneous treatment effects---treatment effects vary across patients---and,
as such, may not be applicable to most patient subgroups, let alone individual
patients. If a treatment causes a serious adverse event, then treating all
patients on the basis of an observed overall positive treatment effect may be
harmful for some {[}Rothwell, Lancet 1995; \ldots{]}.

Conceptually, heterogeneity of treatment effect (HTE) is the variation of
treatment effects on the individual level within the population
{[}\url{https://doi.org/10.1111/j.0887-378X.2004.00327.x}{]}. The identification and
quantification of HTE is crucial for guiding medical decision making and lies at the
core of patient-centered outcomes research. Despite HTE being widely
anticipated, however, its evaluation is not straightforward. Individual
treatment effects are---by their nature---unobservable: the moment a patient
receives a specific treatment, their response under the alternatives becomes
unmeasurable (fundamental problem of causal inference; {[}Holland, 1986{]}).

To ``glimpse'' at a specific individual's response under alternative treatments,
researchers usually observe the outcomes of other ``similar'' patients that
actually received one of the other candidate treatments. More individualized
treatment effects are derived from the average effects estimated within a
subgroup of similar patients. However, patient similarity is not straightforward
to assess. Patients differ in a vast number of characterstics which may or may
not be relevant to modifying treatment responses. Identification of such patient
characterstics can be quite complicated. In clinical trials it usually relies on
the detection of statistically significant interactions of treatment with
measured covariates (subgroup analyses). However, as clinical trials are usually
only adequately powered to detect overall effects of a certain size, this kind
of analyses can be problematic. This is already widely known and research in
subgroup analyses has provided guidance on how these should be carried out
{[}refs{]}. The lack of statistical power often results in falsely concluding
``consistency'' of the treatment effect across several subpopulations of interest
or overestimating the effect size of a treatment-covariate interaction. The
former because an existing interaction effect was smaller than the detectable
effect size, the latter because of false positives introduced from multiple
testing.

Baseline risk, i.e.~the probability---given measured characteristics---of
experiencing the outcome of interest without receiving the treatment under
study, is an important determinant of treatment effect {[}refs{]}. It sets an upper
bound on the treatment effect size. Low risk patients can only experience
minimal treatment benefit before their risk is reduced to zero, while high risk
patients can benefit much more. This means that baseline risk can be used as a
subgrouping variable for assessing HTE. For many common settings prediction
models of high quality for estimating baseline risk already exist and can be
directly applied to the data at hand {[}refs{]}. If no such models exist, the
researcher can develop one from the available dataset {[}refs{]}.

Baseline risk can also be used for directly predicting individual treatment
benefit {[}Califf; Dahabreh, IJE 2016{]}. For example Califf et al {[}ref{]} predicted
individual benefits regarding mortality with tissue plasminogen activator (tPA)
compared to streptokinase treatment in patients with acute myocardial infarction
using baseline mortality risk and assuming a constant relative tPA treatment
effect. However, relative treatment effect does not need to be assumed
constant. Modeling more flexible interactions of treatment with baseline outcome
risk may provide more informative absolute benefit predictions for individual
patients.

Depending on the scale treatment effect is measured, HTE may or may not be
identified. For example, despite finding statistically significant subgroup
effect evaluated on the relative scale, the absolute risk difference between the
two groups may be so small that has no clinical relevance
{[}\url{http://dx.doi.org/10.1016/j.jclinepi.2013.11.003}{]}. Therefore, in the presence
of a trully effective treatment, effect heterogeneity should always be
anticipated on some scale {[}Dahabreh, IJE 2016{]}, as baseline risk is bound to
vary across trial patients. If effect modifiers are known and the available
sample size provides adequate statistical power for evaluating
treatment-covariate interactions, modeling these interactions would be the
optimal approach for assessing HTE. However, this approach may lead to
overfitting and unstable estimates for the interaction effects {[}Balan, JCE{]}.

Healthcare data is routinely collected by general practitioners, hospitals,
insurance companies, and many other private or public bodies and is becoming
increasingly available giving researchers access to massive amounts of patient
data. Theoretically, the aforementioned statistical power challenges for the
evaluation of HTE would be largely mitigated if the analyses were performed on
even a single such database. However, as this data is not being accumulated for
research purposes, it suffers from many biases causing many commonly used
methods to fail. Doctors prescribing a specific treatment expect---usually
based on results from clinical trials---that it will be beneficial for the
patient they are treating. This causes systematic differences in important
characteristics among patients receiving different treatments and renders their
comparison very challenging.

If all relevant patient characteristics on which the treating physician based
their decision have been captured in the observational dataset, methods are
available that can be used to account for these systematic differences
{[}refs{]}. Among the more popular ones is limiting the analyses to the propensity
score matched subpopulation. Propensity scores are the patient-specific
probabilities of receiving the treatment under study and have been shown to have
the balancing property, i.e.~conditional on the propensity score treatment
assignment is independent of the potential outcomes {[}refs{]}. This means that in a
subset of patients with equal propensity scores there are no differences in
covariate distributions between patients receiving the treatment under study and
those who are not. Consequently, patients within this subset can be assumed to
be randomized.

Unfortunately, not all the information that was used to decide on treatment is
captured. As a consequence, propensity score adjustment will not suffice to
evaluate treatment effects using the observational data, be it overall or
subgroup effects. Sensitivity analyses searching for evidence of this systematic
unmeasured imbalances have been proposed and can be of assistance in many
situations {[}refs{]}.

Another important problem with observational databases is that they are not
compatible with each other. As anyone gathering routinely observed healthcare
data did so in a way that was more convenient to them, a plethora of structures
for the resulting databases arose. Diseases, treatments, medical exams and many
more aspects of healthcare are often coded differently in different
observational databases. In addition, more fundamental disparities between
databases also factor in database incopatibility: different types of information
are recorded in different databases. Different patient characteristics are
captured---at different levels of detail---in a general practitioner database,
in a hospital medical record or in an administrative claims database. This means
that combining results from multiple databases is not a simple task.

One of the solutions put forward for handling database incompatibility was the
creation of the Observational Medical Outcomes Partnership Common Data Model
(OMOP-CDM) {[}refs{]}. This provided a standard for structuring an observational
database while large effort was put into developing processes for mapping
existing databases from their own specific structure to OMOP-CDM. A high level of
standardization and scalability for observational studies was achieved. Common
definitions of diseases, treatments, and outcomes can now be applied uniformly
across a network of many databases containing information on hundreds of
millions of patients. An analysis plan can be executed following the exact same
steps across the network providing effect estimates derived in different
populations. The fragmented information scattered across multiple databases can
now be summarized in a consistent way to give a fuller picture.

The power of the common database structure was demonstrated in a large-scale
comparative effectiveness study of first-line treatment for hypertension
{[}Suchard, Lancet 2019{]}. This study compared five different first-line drug
classes prescribed for hypertension regarding three primary effectiveness, six
secondary effectiveness, and 46 safety outcomes across a global network of 9
observational databases, all mapped to OMOP-CDM. A framework following best
practices for carrying out such analyses was proposed and implemented on a large
scale. The results complemented the already available evidence generated in
clinical trials, confirming earlier findings and providing effect estimates on
previously unexplored comparisons.

Observational databases provide access to massive numbers of ``real-life''
patients. This motivates the exploration of methods for the assessment of
treatment effect heterogeneity in the observational setting despite the
challenges inherent to this type of data. The statistical power problems related
to subgroup analyses can still be present, as observational data is
high-dimensional, i.e.~the number of measured patient characteristics increases
with the number of patients. Attempting a treatment effect modeling approach,
where treatment-covariate interactions are modeled for the prediction of
individualized treatment benefits, suffers from the same statistical power
issues and often results to highly variable estimates. Therefore, using baseline
outcome risk as the subgrouping variable, can provide good insight of treatment
effect heterogeneity in the observational setting, as well. Modern libraries for
developing risk prediction models are available and---capitalizing on
OMOP-CDM---can be easily applied across databases with millions of patients.

The overall aim of this thesis is to explore approaches that incorporate
baseline risk as the basis for medical decision making, shifting the focus from
outcome prediction to the evaluation treatment effect heterogeneity. We will
explore methods and applications in both the clinical trial and the
observational setting. More specifically, the main research questions are:

\begin{itemize}
\tightlist
\item
  Aim 1: \emph{How to develop and present a prediction model to be used in clinical
  practice?} We will explore the development of multi-purpose prediction models,
  i.e.~models that can be used to predict patient risks for multiple
  outcomes. We will also demonstrate different approaches for their application
  in practice.
\item
  Aim 2: \emph{How to use risk estimates from a prediction model to assess treatment
  effect heterogeneity?} We will explore the literature for methods on the
  assessment of treatment effect heterogeneity. We will develop and apply a
  framework for risk-based assessment of treatment effect heterogeneity in the
  observational setting. Finally, we will explore methods for making
  individualized risk-based benefit predictions.
\end{itemize}

In \textbf{Chapter 2} we develop a model for the prediction of 5-year recurrence risk
in sentinel node positive melanoma patients, using data from nine European
Organization for Research and Treatment of Cancer centers. We calibrate the
recurrence model to predict 5-year risk of distant metastasis and overall
mortality and develop a nomogram for graphical representation. The models are,
then, extenrally validated.

In \textbf{Chapter 3} we develop a model for the prediction of 28-day mortality for
patients presenting at the emergency department with suspected COVID-19
infection at four large Dutch hospitals between March and August, 2020. We
predict 28-day admission to the intensive care unit by calibrating the mortality
model. An easy to use web application is also supplied. We perform temporal
validation to assess model performance using data between September and
December, 2020.

In \textbf{Chapter 4} we present the results of a scoping literature review of
regression modeling approaches for the assessment of treatment effect
heterogeneity in the clinical trial setting. The identified methods are divided
into broader categories based on how they incorporate prognostic factors and
treatment effect modifiers.

In \textbf{Chapter 5} we develop a standardized scalable framework for the assessment
of treatment effect heterogeneity using a risk-stratified approach in the
observational setting. We, also, develop the software for the execution of the
framework in observational databases mapped to OMOP-CDM. We, finally,
demonstrate the application of the framework, assessing treatment effect
heterogeneity in first-line treatment for hypertension across three US claims
databases.

In \textbf{Chapter 6} we apply the standardized framework to evaluate effect
heterogeneity of teripatide treatment compared to oral bisphosphonates in female
patients above the age of 50 with established osteoporosis. We use different
risk stratification approaches based on quantiles of predicted risk and
externally derived risk thresholds for treatment. We evaluate the presence of
residual confounding using sensitivity analyses.

In \textbf{Chapter 7} we compare different risk-based methods for predicting
individualized treatment effects using an extensive simulation study. We only
consider the clinical trial setting were treatment is admistered at random.

Finally, in \textbf{Chapter 8} we present a general discussion along with perspectives on
future work.

\end{document}
