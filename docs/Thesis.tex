% Options for packages loaded elsewhere
\PassOptionsToPackage{unicode}{hyperref}
\PassOptionsToPackage{hyphens}{url}
%
\documentclass[
]{book}
\usepackage{amsmath,amssymb}
\usepackage{lmodern}
\usepackage{iftex}
\ifPDFTeX
  \usepackage[T1]{fontenc}
  \usepackage[utf8]{inputenc}
  \usepackage{textcomp} % provide euro and other symbols
\else % if luatex or xetex
  \usepackage{unicode-math}
  \defaultfontfeatures{Scale=MatchLowercase}
  \defaultfontfeatures[\rmfamily]{Ligatures=TeX,Scale=1}
\fi
% Use upquote if available, for straight quotes in verbatim environments
\IfFileExists{upquote.sty}{\usepackage{upquote}}{}
\IfFileExists{microtype.sty}{% use microtype if available
  \usepackage[]{microtype}
  \UseMicrotypeSet[protrusion]{basicmath} % disable protrusion for tt fonts
}{}
\makeatletter
\@ifundefined{KOMAClassName}{% if non-KOMA class
  \IfFileExists{parskip.sty}{%
    \usepackage{parskip}
  }{% else
    \setlength{\parindent}{0pt}
    \setlength{\parskip}{6pt plus 2pt minus 1pt}}
}{% if KOMA class
  \KOMAoptions{parskip=half}}
\makeatother
\usepackage{xcolor}
\usepackage{longtable,booktabs,array}
\usepackage{calc} % for calculating minipage widths
% Correct order of tables after \paragraph or \subparagraph
\usepackage{etoolbox}
\makeatletter
\patchcmd\longtable{\par}{\if@noskipsec\mbox{}\fi\par}{}{}
\makeatother
% Allow footnotes in longtable head/foot
\IfFileExists{footnotehyper.sty}{\usepackage{footnotehyper}}{\usepackage{footnote}}
\makesavenoteenv{longtable}
\usepackage{graphicx}
\makeatletter
\def\maxwidth{\ifdim\Gin@nat@width>\linewidth\linewidth\else\Gin@nat@width\fi}
\def\maxheight{\ifdim\Gin@nat@height>\textheight\textheight\else\Gin@nat@height\fi}
\makeatother
% Scale images if necessary, so that they will not overflow the page
% margins by default, and it is still possible to overwrite the defaults
% using explicit options in \includegraphics[width, height, ...]{}
\setkeys{Gin}{width=\maxwidth,height=\maxheight,keepaspectratio}
% Set default figure placement to htbp
\makeatletter
\def\fps@figure{htbp}
\makeatother
\setlength{\emergencystretch}{3em} % prevent overfull lines
\providecommand{\tightlist}{%
  \setlength{\itemsep}{0pt}\setlength{\parskip}{0pt}}
\setcounter{secnumdepth}{5}
\usepackage{booktabs}
\ifLuaTeX
  \usepackage{selnolig}  % disable illegal ligatures
\fi
\usepackage[]{natbib}
\bibliographystyle{apalike}
\IfFileExists{bookmark.sty}{\usepackage{bookmark}}{\usepackage{hyperref}}
\IfFileExists{xurl.sty}{\usepackage{xurl}}{} % add URL line breaks if available
\urlstyle{same} % disable monospaced font for URLs
\hypersetup{
  pdftitle={Assessing treatment effect heterogeneity using baseline risk},
  pdfauthor={Alexandros Rekkas},
  hidelinks,
  pdfcreator={LaTeX via pandoc}}

\title{Assessing treatment effect heterogeneity using baseline risk}
\usepackage{etoolbox}
\makeatletter
\providecommand{\subtitle}[1]{% add subtitle to \maketitle
  \apptocmd{\@title}{\par {\large #1 \par}}{}{}
}
\makeatother
\subtitle{Methodology and applications}
\author{Alexandros Rekkas}
\date{2022-05-18}

\begin{document}
\maketitle

{
\setcounter{tocdepth}{1}
\tableofcontents
}
\hypertarget{preface}{%
\chapter*{Preface}\label{preface}}
\addcontentsline{toc}{chapter}{Preface}

Thesis

\hypertarget{introduction}{%
\chapter{Introduction}\label{introduction}}

In order to provide---on average---the most current medical care doctors are
advised to align their clinical practice with the results of well-conducted
clinical trials, or the aggregation of the results from multiple such trials
{[}Greenfield 2007{]}. This approach implicitly assumes that all patients eligible
for treatment similarly experience the benefits and harms of treatment of the
reference trial population. Therefore, at a certain point, the accurate
estimation of these average effects became crucial, transforming clinical trials
from tools for assessing causality into tools for predicting patient-level
treatment effects. When the strong positive overall effects derived from
clinical trials could not be achieved in medical practice, the problem was
attributed to the reference trial population being too narrow and not
representing the ``average'' patient requiring treatment. Therefore the need for
more pragmatic clinical trials incorporating broad patient populations was
highlighted {[}Treweek, Trials 2009; \ldots{]}.

The wider clinical trial populations ensure that overall results will be
generalizable to the ``real-life'' patients. However, generalizability comes at a
cost: wider range of included patients means higher variability of measured
characteristics, therefore higher variability in disease severity is observed,
which, in turn, translates to higher variability of observed treatment effect
sizes. In short, the estimated treatment effect derived from such clinical
trials is often an average of heterogeneous treatment effects and, as such, is
not applicable to most patient subgroups. This means that a positive average
treatment effect estimated from a clinical trial very often is only evidence
that some of the enrollees benefitted from the intervention under study. If,
however, the intervention is linked to a serious adverse event, treating
everyone would result in serious harms for many patients, despite the positive
overall effect.

Conceptually, heterogeneity of treatment effects (HTE) is the variation of
treatment effects on the individual level across the the population
{[}\url{https://doi.org/10.1111/j.0887-378X.2004.00327.x}{]}. As individual treatment
effects---difference between outcomes under all possible treatment assignments
within the same individual---are the ones that generate HTE, its identification
and quantification is crucial for guiding medical decision making and lies at
the core of patient-centered research. Despite HTE being widely anticipated,
however, its evaluation is not straightforward. Individual treatment effects
are---by their nature---unobservable since, the moment a patient receives a
specific treatment, their response under the alternatives becomes unmeasurable
(fundamental problem of causal inference).

To ``glimpse'' at a specific individual's response under alternative treatments,
researchers usually observe the outcomes of other ``similar'' patients that did
receive one of the other candidate treatments. However, patient similarity is
not straightforward to assess. Patients differ in a vast number of
characterstics that make them unique and may or may not be relevant to modifying
treatment responses. Identification of such patient characterstics can be quite
complicated. In clinical trials it usually relies on the detection of measured
covariates with a statistically significant interaction with treatment. However,
as clinical trials are usually only adequately powered to detect overall effects
of a certain size, this kind of analyses can be quite problematic. This is
already widely known and a large part of research in subgroup analyses has
provided guidance on how these should be carried out {[}refs{]}. This lack of
statistical power often results in falsely concluding ``consistency'' of the
treatment effect across several subpopulations of interest, or overestimating
the effect size of a treatment-covariate interaction. The former because an
existing interaction effect was smaller than the detectable effect size, the
latter because randomization did not achieve balance between the levels of the
subgrouping variable due to the small number of patients belonging to the
specific subgroup.

they require
knowledge of patient-level outcomes under all possible treatment assignments
(fundamental problem of causal inference). For that reason an average effect
derived from a patient subgroup is often used for making ``individualized''
evaluations. These subgoups are usually defined based on a single patient
characteristic, comparing treatment effects between males and females, older and
younger patients, and any other covariate assumed to be relevant. However, as
clinical trials are most often adequately powered to detect a certain overall
effect size, these subgroup analyses usually are underpowered. This can lead to
falsely claiming absence of HTE or overestimating its magnitude {[}refs{]}. In
addition, contrary to subgroup analyses, patients differ with regard to many
baseline covariates simultaneously {[}Kent, BMJ 2018{]}. However, evaluation of
two-way or higher order interactions becomes underpowered very fast.

Kravitz et al {[}\url{https://doi.org/10.1111/j.0887-378X.2004.00327.x}{]} linked
prediction of individual treatment effect on the knowledge of the patient's
baseline risk, responsiveness to treatment, vulnerability to side-effects, and
the utilities the patient places for different outcomes. Therefore, baseline
risk, i.e.~the patient's true probability of having an outcome of
interest---usually the outcome treatment is attempting to prevent--- without
receiving the treatment under study is already recognized as an important
determinant of treatment effect. It sets an upper bound on how large the
treatment effect can be for individuals of the same risk. Low risk patients can
only experience minimal treatment benefit before their risk is reduced to
zero. On the other hand, high risk patients can achieve much higher
benefits. But, depending on the scale treatment effect is measured, HTE may be
present or absent. For example, despite finding statistically significant
subgroup effect evaluated on the relative scale, the absolute risk difference
between the two groups may be so small that has no clinical relevance
{[}\url{http://dx.doi.org/10.1016/j.jclinepi.2013.11.003}{]}. Therefore, in the presence
of a trully effective treatment, effect heterogeneity should always be
anticipated on some scale {[}Dahabreh, IJE 2016{]}, as baseline risk is bound to
vary across trial patients.

In the presence of true effect modification, accurately modeling
treatment-covariate interactions would result in adequate evaluation of
HTE. However, simulations have shown that due to the low power for correctly
estimating these interactions, these approaches require very large sample
sizes. Modeling effect modifiers essentially suffers from the same issues as
subgroup analyses.

Healthcare data routinely collected by general practitioners, hospitals,
insurance companies, and many other private or public bodies is becoming
increasingly available for research, giving researchers access to massive
amounts of patient data. Theoretically, the aforementioned power issues for the
evaluation of HTE would be largely mitigated if the analyses were performed on
even a single observational database. However, as this data is not being
accumulated with research in mind, it suffers from many biases causing most of
the traditional inference methods to fail. The main source of all the problems
is that observational data is not randomized. A doctor prescribing a specific
treatment knows---usually based on results from RCTs---that it will be
beneficial for the patient they are treating. This often causes patients of
similar characteristics following the same treatment course, resulting in
non-random differences between the several treatments. This makes comparisons of
treatments quite difficult to evaluate.

\end{document}
