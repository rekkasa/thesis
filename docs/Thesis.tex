% Options for packages loaded elsewhere
\PassOptionsToPackage{unicode}{hyperref}
\PassOptionsToPackage{hyphens}{url}
%
\documentclass[
]{book}
\usepackage{amsmath,amssymb}
\usepackage{lmodern}
\usepackage{iftex}
\ifPDFTeX
  \usepackage[T1]{fontenc}
  \usepackage[utf8]{inputenc}
  \usepackage{textcomp} % provide euro and other symbols
\else % if luatex or xetex
  \usepackage{unicode-math}
  \defaultfontfeatures{Scale=MatchLowercase}
  \defaultfontfeatures[\rmfamily]{Ligatures=TeX,Scale=1}
\fi
% Use upquote if available, for straight quotes in verbatim environments
\IfFileExists{upquote.sty}{\usepackage{upquote}}{}
\IfFileExists{microtype.sty}{% use microtype if available
  \usepackage[]{microtype}
  \UseMicrotypeSet[protrusion]{basicmath} % disable protrusion for tt fonts
}{}
\makeatletter
\@ifundefined{KOMAClassName}{% if non-KOMA class
  \IfFileExists{parskip.sty}{%
    \usepackage{parskip}
  }{% else
    \setlength{\parindent}{0pt}
    \setlength{\parskip}{6pt plus 2pt minus 1pt}}
}{% if KOMA class
  \KOMAoptions{parskip=half}}
\makeatother
\usepackage{xcolor}
\usepackage{longtable,booktabs,array}
\usepackage{calc} % for calculating minipage widths
% Correct order of tables after \paragraph or \subparagraph
\usepackage{etoolbox}
\makeatletter
\patchcmd\longtable{\par}{\if@noskipsec\mbox{}\fi\par}{}{}
\makeatother
% Allow footnotes in longtable head/foot
\IfFileExists{footnotehyper.sty}{\usepackage{footnotehyper}}{\usepackage{footnote}}
\makesavenoteenv{longtable}
\usepackage{graphicx}
\makeatletter
\def\maxwidth{\ifdim\Gin@nat@width>\linewidth\linewidth\else\Gin@nat@width\fi}
\def\maxheight{\ifdim\Gin@nat@height>\textheight\textheight\else\Gin@nat@height\fi}
\makeatother
% Scale images if necessary, so that they will not overflow the page
% margins by default, and it is still possible to overwrite the defaults
% using explicit options in \includegraphics[width, height, ...]{}
\setkeys{Gin}{width=\maxwidth,height=\maxheight,keepaspectratio}
% Set default figure placement to htbp
\makeatletter
\def\fps@figure{htbp}
\makeatother
\setlength{\emergencystretch}{3em} % prevent overfull lines
\providecommand{\tightlist}{%
  \setlength{\itemsep}{0pt}\setlength{\parskip}{0pt}}
\setcounter{secnumdepth}{5}
\usepackage{booktabs}
\ifLuaTeX
  \usepackage{selnolig}  % disable illegal ligatures
\fi
\usepackage[]{natbib}
\bibliographystyle{apalike}
\IfFileExists{bookmark.sty}{\usepackage{bookmark}}{\usepackage{hyperref}}
\IfFileExists{xurl.sty}{\usepackage{xurl}}{} % add URL line breaks if available
\urlstyle{same} % disable monospaced font for URLs
\hypersetup{
  pdftitle={Assessing treatment effect heterogeneity using baseline risk},
  pdfauthor={Alexandros Rekkas},
  hidelinks,
  pdfcreator={LaTeX via pandoc}}

\title{Assessing treatment effect heterogeneity using baseline risk}
\usepackage{etoolbox}
\makeatletter
\providecommand{\subtitle}[1]{% add subtitle to \maketitle
  \apptocmd{\@title}{\par {\large #1 \par}}{}{}
}
\makeatother
\subtitle{Methodology and applications}
\author{Alexandros Rekkas}
\date{2022-05-15}

\begin{document}
\maketitle

{
\setcounter{tocdepth}{1}
\tableofcontents
}
\hypertarget{preface}{%
\chapter*{Preface}\label{preface}}
\addcontentsline{toc}{chapter}{Preface}

Thesis

\hypertarget{introduction}{%
\chapter*{Introduction}\label{introduction}}
\addcontentsline{toc}{chapter}{Introduction}

In order to provide---on average---the most current medical care doctors are
advised to align their clinical practice with the results of well-conducted
clinical trials, or the aggregation of the results from multiple such trials
{[}Greenfield 2007{]}. This approach implicitly assumes that all patients eligible
for treatment similarly experience the benefits and harms of treatment of the
reference trial population. Therefore, at a certain point, the accurate
estimation of these average effects became crucial, transforming clinical trials
from tools for assessing causality into tools for predicting patient-level
treatment effects. When the strong positive overall effects derived from
clinical trials could not be achieved in medical practice, the problem was
attributed to the reference trial population being too narrow and not
representing the ``average'' patient requiring treatment. Therefore the need for
more pragmatic clinical trials incorporating broad patient populations was
highlighted {[}Treweek, Trials 2009; \ldots{]}.

The wider clinical trial populations ensure that overall results will be
generalizable to the ``real-life'' patients. However, generalizability comes at a
cost: wider range of included patients means higher variability of measured
characteristics, therefore higher variability in disease severity is observed,
which, in turn, translates to higher variability of observed treatment effect
sizes. In short, the estimated treatment effect derived from such clinical
trials is often an average of heterogeneous treatment effects and, as such, is
not applicable to most patient subgroups. This means that a positive average
treatment effect estimated from a clinical trial very often is only evidence
that some of the enrollees benefitted from the intervention under study. If,
however, the intervention is linked to a serious adverse event, treating
everyone would result in serious harms for many patients, despite the positive
overall effect.

The identification and quantification of treatment effect heterogeneity (HTE),
i.e.~the non-random variability in the magnitude or direction of a treatment
effect across the levels of a covariate (single or combination of patient
attributes) {[}refs{]}, is crucial for guiding medical decision-making. Despite HTE
being widely anticipated, its evaluation is not straightforward. At the
individual level HTE is unobservable as it requires knowledge of patient-level
outcomes under all possible treatment assignments (fundamental problem of causal
inference). The most common solution is, then, to assign the patient to a
subgroup with similar anticipated treatment effect, assigning the subgroup-level
effect estimate to the individual. Subgroup analyses have often been considered
for the evaluation of HTE, where the clinical trial population is split into
sub-populations based on the levels of a certain covariate, one at a
time. However, as clinical trials are most often adequately powered to detect a
certain overall effect size, these subgroup analyses usually are
underpowered. This can lead to falsely claiming absence of HTE or overestimating
its magnitude {[}refs{]}. In addition, contrary to subgroup analyses, patients
differ with regard to many baseline covariates simultaneously {[}Kent, BMJ
2018{]}. However, evaluation of two-way or higher order interactions becomes
underpowered at an explonential rate.

Kravitz et al {[}\url{https://doi.org/10.1111/j.0887-378X.2004.00327.x}{]} linked
prediction of individual treatment effect on the knowledge of the patient's
baseline risk, responsiveness to treatment, vulnerability to side-effects, and
the utilities the patient places for different outcomes. Therefore, baseline
risk, i.e.~the patient's true probability of having an outcome of
interest---usually the outcome treatment is attempting to prevent---is already
recognized as an important determinant of treatment effect. It sets an upper
bound on how much treatment can have an effect on the individual. Low risk
patients can only at most observe minimal treatment benefit before their risk is
reduced to zero. On the other hand, high risk patients can achieve much higher
benefits. Therefore, in the presence of a trully effective treatment, we would
expect increasing absolute benefits. However, this is bound to be the case, in
the absence of true effect modification.

\end{document}
